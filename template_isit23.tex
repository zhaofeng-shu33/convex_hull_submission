%% LaTeX Template for ISIT 2023
%%
%% by Stefan M. Moser, June 2022
%% 
%% derived from bare_conf.tex, V1.4a, 2014/09/17, by Michael Shell
%% for use with IEEEtran.cls version 1.8b or later
%%
%% Support sites for IEEEtran.cls:
%%
%% https://www.michaelshell.org/tex/ieeetran/
%% https://moser-isi.ethz.ch/manuals.html#eqlatex
%% https://www.ctan.org/tex-archive/macros/latex/contrib/IEEEtran/
%%

\documentclass[conference,a4paper]{IEEEtran}


%% depending on your installation, you may wish to adjust the top margin:
\addtolength{\topmargin}{9mm}
%% apart from this
%% *** do not adjust lengths that control margins, column widths, etc.! ***
%% *** do not use packages that alter fonts (such as pslatex)!          ***

%%%%%
%% Packages:
\usepackage[utf8]{inputenc} 
\usepackage[T1]{fontenc}
\usepackage{url}              % provides \url{...}
%\usepackage{ifthen}          % provides \ifthenelse
\usepackage{cite}             % improves presentation of citations

\usepackage[cmex10]{amsmath}  % Use the [cmex10] option to ensure complicance
                              % with IEEEXplore (see bare_conf.tex)
\interdisplaylinepenalty=1000 % As explained in bare_conf.tex
\usepackage{mleftright}       % fix to wrong spacing of \left-,
\mleftright                   % \middle- \right-commands 

\usepackage{graphicx}         % provides \includegraphics{...} to
                              % include graphics (pdf format)
\usepackage{booktabs}         % fixes poor spacing in tables and
                              % provides \toprule, \midrule, \bottomrule

%\usepackage{algorithmicx}    % provides an algorithmic environment for
                              % describing algorithms. See
                              % https://ctan.org/pkg/algorithmicx

% \usepackage[caption=false,font=footnotesize]{subfig}
                              % provides subnumbering within a
                              % floating figure or table

%% For arrays and multiple-line equations, use the
%% IEEEeqnarray-environment. See
%%              https://moser-isi.ethz.ch/manuals.html#eqlatex  
%% for instructions.

%% Do NOT use amsthm or hyperref!
%% -IEEEtran provides its own versions of theorems.
%% -IEEEXplore does not accept submissions with hyperlinks


%%%%%
%% correct bad hyphenation here
\hyphenation{op-tical net-works semi-conduc-tor}
\usepackage{amsfonts}
\DeclareMathOperator{\dist}{dist}
\def\E{\mathbb{E}}
\def\R{\mathbb{R}}
\def\d{\mathrm{d}}
% -------------------------------------------------------------------------
\begin{document}

\title{On the Expected Number of Facets for the Convex Hull of Samples from Spherically Symmetric Distributions} 

%%%%%%
\author{%
  \IEEEauthorblockN{Anonymous Authors}
  %\IEEEauthorblockA{%
  %  Please do NOT provide authors' names and affiliations\\
  %  in the paper submitted for review, but keep this placeholder.\\
  %  ISIT23 follows a \textbf{double-blind reviewing policy}.}
}

%%%%%% Please only add the author names and affiliations for the FINAL
%%%%%% version of the paper, but NOT for the paper submitted for review!
%
%%%%%
%%%%% Single author, or several authors with same affiliation:
% \author{%
%   \IEEEauthorblockN{Stefan M.~Moser}
%   \IEEEauthorblockA{ETH Zürich\\
%                     8092 Zürich, Switzerland\\
%                     moser@isi.ee.ethz.ch}
%                   }
%
%%%%%
%%%%% Several authors with up to three affiliations:
% \author{%
%   \IEEEauthorblockN{Stefan M.~Moser}
%   \IEEEauthorblockA{ETH Zürich\\
%                     ISI (D-ITET), ETH Zentrum\\
%                     8092 Zürich, Switzerland\\
%                     moser@isi.ee.ethz.ch}
%   \and
%   \IEEEauthorblockN{Albus Dumbledore and Harry Potter}
%   \IEEEauthorblockA{Hogwarts School of Witchcraft and Wizardry\\
%                     Hogwarts Castle\\ 
%                     1714 Hogsmeade, Scotland\\
%                     \{dumbledore, potter\}@hogwarts.edu}
% }
%
%%%%%   
%%%%% Many authors with many affiliations:
% \author{%
%   \IEEEauthorblockN{Albus Dumbledore\IEEEauthorrefmark{1},
%                     Olympe Maxime\IEEEauthorrefmark{2},
%                     Stefan M.~Moser\IEEEauthorrefmark{3}\IEEEauthorrefmark{4},
%                     and Harry Potter\IEEEauthorrefmark{1}}
%   \IEEEauthorblockA{\IEEEauthorrefmark{1}%
%                     Hogwarts School of Witchcraft and Wizardry,
%                     1714 Hogsmeade, Scotland,
%                     \{dumbledore, potter\}@hogwarts.edu}
%   \IEEEauthorblockA{\IEEEauthorrefmark{2}%
%                     Beauxbatons Academy of Magic,
%                     1290 Pyrénées, France,
%                     maxime@beauxbatons.fr}
%   \IEEEauthorblockA{\IEEEauthorrefmark{3}%
%                     ETH Zürich, ISI (D-ITET), ETH Zentrum, 
%                     CH-8092 Zürich, Switzerland,
%                     moser@isi.ee.ethz.ch}
%   \IEEEauthorblockA{\IEEEauthorrefmark{4}%
%                     National Yang Ming Chiao Tung University (NYCU), 
%                     Hsinchu, Taiwan,
%                     moser@isi.ee.ethz.ch}
% }
%

\maketitle

%%%%%
%% Abstract: 
%% If your paper is eligible for the student paper award, please add
%% the comment "THIS PAPER IS ELIGIBLE FOR THE STUDENT PAPER
%% AWARD." as a first line in the abstract. 
%% For the final version of the accepted paper, please do not forget
%% to remove this comment!
%%
\begin{abstract}
  This paper estimates the sample complexity for efficient learning in high dimensional space.
  To be specific, we model the learning task as interpolation of convex hull consisting of i.i.d. sampled data
  , and the sample complexity is the number of training data which make the probability measure of the convex hull
  tend to one. It is shown that the sample complexity has exponential relationship with the dimension parameter for all the distribution families
  considered in this paper, which gives insight on why high dimensional learning is difficult in a general sense. 
\end{abstract}


\section{Introduction}
\label{sec:intro}
In machine learning, intuitively we model a learning task as interpolating the
training data.
Balestriero et al. \cite{balestriero2021learning}
argues that interpolation almost surely never occurs in high-dimensional spaces.
Therefore, we can not think machine learning algorithms work well because they can interpolate training data
well.
Following the idea of Balestriero, we can further enhance this argument using the probability $p_{N,d}$ introduced previously.
The randomly sampled points in $\R^d$ represent training data
while the interpolation
symbolizes the learning algorithm. Then $p_{N,d}$ represents
the probability of region not learnt by the algorithm.
When we expect $p_{N,d} \to 0$ for large $d$,
in this section
we find that we need
at least exponentially large $N$,
which is impractical for real-world dataset.


Let $X_1, X_2, \dots, X_N$ be i.i.d. random points generated from
a spherically symmetric distribution in $\mathbb{R}^d$.
For the convex hull $\mathrm{H}_N$ of these $N$ points, we study the number of its facets $F_N$,
where a facet of a $d$-dimensional object is one of its $(d-1)$-dimensional faces.
In this paper, we study the mathematical expectation
of $F_N$, denoted as $\E[F_N]$, and derive its asymptotic value as $N\to \infty$.

Our motivation to study $\E[F_N]$ is to give an upper bound of
the probability $p_{N,d}=P(X_{N+1} \not\in \mathrm{H}_N)$,
which is the probability that the $(N+1)$-th point falls outside the convex hull.
If we accept the concept that interpolation occurs when $X_{N+1}$ belongs to $H_N$,
$p_{N,d}$ can be applied to explain
why interpolation almost surely never occurs in high dimensional space \cite{balestriero2021learning}.
In other words, for a large $d$, $p_{N,d}$ is near $1$, unless exponentially large sample size $N$ is available.
Using the asymptotic result on $\E[F_N]$, we obtain
a sufficient condition under which the interpolation almost surely occurs.


The asymptotic expression of $\E[F_N]$ as $N\to \infty$
was studied initially by R{\'e}nyi and Sulanke \cite{renyi1963konvexe}.
They considered
bivariate Gaussian distribution and uniform distribution
in a planar convex region.
Later on,  R{\'e}nyi's work was generalized by
Carnal \cite{carnal1970konvexe}, who
classified symmetric 2-D distributions
into three categories according to their tails:
polynomial, exponential, and truncated tails.
Then he obtained the asymptotic expression of $\E[F_N]$
for each category of distributions.




The study of $\E[F_N]$ for $d>2$ was made firstly by
Raynaud
\cite{raynaud1970enveloppe}.
He obtained the asymptotic formula of $\E[F_N]$
for uniform distribution in a hyperball
and standard Gaussian distribution in $\mathbb{R}^d$.
Afterwards, following Carnal, Dwyer \cite{dwyer1991convex}
estimated the order of $\E[F_N]$ about $N$
for three different distribution families.

Dwyer's work on the estimation of $\E[F_N]$ only captured its relationship with $N$ but ignored its
dependence on $d$. To our best knowledge, previous works have not revealed the general asymptotic expression of $\E[F_N]$ in $\R^d$.
In Section \ref{sec:int_f}, we develop a method of obtaining $\E[F_N]$ by generalizing Carnal's integration formula
% of $H(x)=P(d_{12}\geq x)$ 
from $d=2$ to higher dimensions.
In Section \ref{sec:three_distriutions}, we then derive the explicit asymptotic
expressions of $\E[F_N]$ for three different distribution families.
In Section \ref{sec:sample_complexity}, based on the asymptotic result of $\E[F_N]$,
we provide a sufficient condition for the convergence of $p_{N,d}$ to zero as $N,d \to \infty$.
The main contribution of this paper is to provide a complete integration formula for $\E[F_N]$,
and following the classification of distributions of previous authors \cite{carnal1970konvexe,dwyer1991convex},
we derive the asymptotic expression
of $\E[F_N]$ for each category.

Below we define some notations to be used throughout this paper.
Without specific emphasis, all $d$-dimensional distributions considered in this paper are spherically symmetric.
Let $X=(X^{(1)},\dots, X^{(d)})$ follow such a distribution.
Then we use $F(x)=P(|X|\geq x)$ to denote the probability that a random point lies outside
a $d$-sphere ($d$-dimensional sphere centered at origin) with radius $x$.
The marginal distribution of each component of $X$ is determined by $G(x)=P(X^{(1)}\geq x)$.
Let $dist$ represent the distance from the origin to the hyperplane spanned by $X_1, \dots, X_d$.
Specifically, in 2-D, the hyperplane is the straight line passing through $X_1$ and $X_2$.
We use $H(x)=P(\dist\geq x)$ to denote the probability that $\dist$ is larger than $x$.
When $n$ tends to infinity, the functions $f(n)$ and $g(n)$ are asymptotic equivalent if $\lim_{n\to \infty} \frac{f(n)}{g(n)}=1$.
We denote this relationship as $f(n) \sim g(n)$.

\subsection{Detailed Instructions for Double-Blind Submission}
\label{sec:deta-instr}

\begin{itemize}
\item Remove the names and affiliations of authors from the title
  page.
\item Remove acknowledgments.
\item Remove project titles or names that could be used to trace back
  to the authors via web search.  
\item Carefully name your files to anonymize author information.
\item Carefully refer to related work, particularly your own. Do not
  omit references to provide anonymity, as this leaves the reviewer
  incapable of grasping the context. Instead, reference your past work
  in the third person, just as you would any other piece of related
  work by another author. For example, instead of ``In prior work [1],
  we presented a scheme that\ldots,'' sentences in the spirit of ``In
  prior work, Moser \emph{et al.} [1] presented a scheme that\ldots''
  should be used. In this way, the full citation of the referred paper
  can still be given, such as ``[1] S. M. Moser, \ldots''.  It is not
  acceptable to say ``[1] Reference deleted for double-blind review.''
\item We recommend to delay the posting of the submitted manuscript or
  its title/abstract on a public website, such as arXiv or on public
  mailing lists, until after the reviewing process is completed. (This
  is not enforced)
\item The submitted manuscript (PDF file) should be
  text-searchable. Any submission that does not meet this requirement
  may be returned without review.
\item Many of the editing tools automatically add metadata to the
  generated PDF file containing information that may violate the
  double-blind policy. Please remove any possible metadata that can
  link your manuscript to you. This includes removing names,
  affiliation, license numbers etc. from the metadata as well as from
  the paper. Failing to meet this requirement may also lead to a
  rejection without review.
\end{itemize}

Contact the TPC chairs if you have any questions regarding the
double-blind reviewing policy. 



\section{Page Limit}
\label{sec:page-limit}

The main body of a manuscript should not exceed \textbf{5 pages} in
length. There can be an \textbf{optional extra page} containing
references only. Moreover, the manuscript submitted for reviewing may
contain an \textbf{optional 5-page appendix} with additional proofs
and details that may be helpful for the reviewers. This appendix will
have to be removed in the final version of an accepted paper.

Summary: \textbf{5+1+5 in double-column format}.


\section{Paper Format}
\label{sec:paper-format}

\subsection{Template}
\label{sec:template}

The paper must be formatted as shown in this sample
\LaTeX{}-file. More help regarding \LaTeX{} can be found in
\cite{Laport:LaTeX, GMS:LaTeXComp, oetiker_latex, typesetmoser,
  shell15}. Note that the source of this file relies on IEEEtran.cls,
which is part of any modern \LaTeX{}-distribution and can be found on
\url{https://www.ctan.org/tex-archive/macros/latex/contrib/IEEEtran/}

The use of a text processing system other than \LaTeX{} is not
recommended. In particular, we very strongly discourage the use of
Microsoft Word.  Users of any text processing system other than
\LaTeX{} should attempt to duplicate the style of this example as
closely as possible (including font type and size).


\subsection{Formatting}
\label{sec:formatting}

The paper must be A4 or letter size and in double-column format.  The
style of references, equations, figures, tables, etc., should be the
same as for the \emph{IEEE Transactions on Information Theory}.

The source file of this template paper contains many more instructions
on how to format your paper. So, example code for different numbers of
authors, for figures and tables, and references can be found there
(they are commented out).

%%%%%%
%% An example of a floating figure using the graphicx package.
%% Note that \label must occur AFTER (or within) \caption.
%% For figures, \caption should occur after the \includegraphics.
%%
% \begin{figure}[htbp]
%   \centering
%   \includegraphics[width=0.3\textwidth]{myfigure}
%   % where a .pdf suffix will be assumed for pdflatex
%   \caption{Simulation results.}
%   \label{fig:sim}
% \end{figure}
%%%%%%

%%%%%%
%% An example of a double column floating figure using two subfigures.
%% (The subfig package must be loaded for this to work; uncomment the
%% corresponding line in the document header.)  The *-version of
%% figure makes the figure use both columns. The subfigure \label
%% commands are set within each subfloat command, the \label for the
%% overall figure must come after \caption.  \hfil must be used as a
%% separator to get equal spacing.
%%
% \begin{figure*}[htbp]
%   \centering
%   \subfloat[Caption of first subfigure.]{%
%     \label{fig:subfigA}
%     \includegraphics[width=4cm]{subfigcase1}}
%   \hfil
%   \subfloat[Caption of second subfigure.]{%
%     \label{fig:subfigB}
%     \includegraphics[width=4cm]{subfigcase1}}  
%   \caption{Two figures showing something.}
%   \label{fig:subfigures}
% \end{figure*}
% We refer to the Figures~\ref{fig:subfigA} and \ref{fig:subfigB} within
% Figure~\ref{fig:subfigures}.
%%%%%% 

%%%%%%
%% An example of a floating table.
%% Note that the \caption command should come BEFORE the table. Table
%% text will default to \footnotesize as IEEE normally uses this
%% smaller font for tables.  The \label must come after \caption as
%% always. The following example relies on booktabs.sty to fix poor
%% spacing within tables and using \toprule, \midrule, and \bottomrule
%% as proper replacement for the corresponding \hline
%%
% \begin{table}[htbp]
%   \centering
%   \caption{Recursion given by Lloyd's algorithm for the case of a
%     Gaussian random variable represented by four values.}
%   \label{tab:gaussfour}
%   \begin{IEEEeqnarraybox}[\IEEEeqnarraystrutmode%
%     % \IEEEeqnarraystrutsizeadd{2pt}{1pt}% uncomment for more spacing
%     ]{l"l"l}
%     \toprule
%     \hat{x}_1 & \hat{x}_2 & \theta \\
%     \midrule
%     0.5 & 1 & 0.75 \\      
%     0.3578 & 1.3288 & 0.8433 \\
%     0.3973 & 1.4011 & 0.8992 \\
%     0.4202 & 1.4450 & 0.9326 \\
%     0.4336 & 1.4714 & 0.9525 \\
%     0.4414 & 1.4872 & 0.9643 \\
%     0.4461 & 1.4966 & 0.9714 \\
%     0.4488 & 1.5022 & 0.9755 \\
%     \vdots && \\
%     0.4528 & 1.5104 & 0.9816 \\
%     \bottomrule
%   \end{IEEEeqnarraybox}
% \end{table}
%%%%%%


For instructions on how to typeset math, especially for multiple-line
equations with broken equations, we refer to \cite{typesetmoser}.

Pages should not be numbered and there should be no footer or header
(both will be added by us during the production of the
proceedings). The authors' affiliation shown in the final version of
accepted papers should constitute a sufficient mailing address for
persons who wish to contact the authors for more details about the
paper.


\subsection{PDF Requirements}
\label{sec:pdf-requirements}

Only electronic submissions in form of a PDF file will be
accepted. The PDF file has to be PDF/A compliant and text-searchable.

A common problem is missing fonts. Make sure that all fonts are
embedded. (In some cases, printing a PDF into a new PDF or creating a
new PDF with Acrobat Distiller may do the trick.) More information is
available from the IEEE website \cite{IEEE:AuthorToolbox}.

%%%%%%
%% See explanation in comments after Conclusion
\enlargethispage{-1.4cm} 
%%%%%%


\section{Submission}
\label{sec:submission}

\subsection{Tracks}
\label{sec:tracks}

This conference features two tracks:
\begin{itemize}
\item For papers in the \textbf{in-person track}, one author must
  present the work personally in Taipei.
\item For papers in the \textbf{virtual track}, the presentations will
  be given live via zoom in front of a mixed virtual and live
  audience, i.e., the in-person attendees of the conference will be
  able to join the virtual sessions live in the conference
  center. Pre-recorded videos will not be accepted.

  The number of papers in the virtual track is limited, and thus a
  slot in the virtual track cannot be guaranteed.
\end{itemize}
In either track, a failure of a live presentation (in-person or via
zoom, respectively) of their work by one of the authors will result in
the removal of the paper from the published proceedings.


\subsection{Submission of Papers for Review}
\label{sec:subm-papers-revi}

Papers in the form of a PDF file, formatted as described above and
following the double-blind policy, may be submitted online at
\begin{center}
  \url{https://edas.info/N29759}
\end{center}
The deadlines for registering and uploading the paper in EDAS is
\begin{itemize}
\item for the virtual track: \textbf{January 3, 2023}, 23:59 Taiwan
  time (UTC+8);
\item for the in-person track: \textbf{January 22, 2023}, 23:59 Taiwan
  time (UTC+8).
\end{itemize}
Each paper must be classified as ``eligible for student paper award''
or ``not eligible for student paper award''. Papers that are selected
to be eligible for the student paper award should also contain
\emph{``THIS PAPER IS ELIGIBLE FOR THE STUDENT PAPER AWARD.''} as a
first line in the abstract. Note that this comment must be removed
again in the final manuscript.

\subsection{Submission of Accepted Papers}
\label{sec:subm-final-vers}

Please do not forget to remove the 5-page appendix if present and to
add authors' names and affiliations and all necessary acknowledgments.
Also remove the sentence \emph{``THIS PAPER IS ELIGIBLE FOR THE
  STUDENT PAPER AWARD.''} in the abstract if present.

The deadline for the submission of the final version of accepted
papers is \textbf{May 6, 2023}, 23:59 Taiwan time (UTC+8).  Accepted
papers not submitted by that date will not appear in the ISIT
proceedings and will not be included in the technical program of the
ISIT.

Please make sure that the meta data in EDAS (list of authors, title
and abstract) match exactly the final version of your paper. For
generating the program booklet, the data in EDAS is used and not the
final version of the paper.


%%%%%
%%% Do not uncomment for double-blind review submission!
% \section*{Acknowledgments}
%
% We are indebted to Michael Shell for maintaining and improving
% \texttt{IEEEtran.cls}. 


\section{Conclusion}
\label{sec:conclusion}

We conclude by pointing out that on the last page the columns need to
be balanced. Instructions for that purpose are given in the source
file (they are commented out).

%%%%%%
%% To balance the columns at the last page of the paper use this
%% command somewhere at the top of the first column of the last page:
%%
% \enlargethispage{-5cm} 
%%
%% where the exact amount of page reduction has to be adapted to the
%% actual situation.
%%
%% If the balancing should occur in the middle of the references, use
%% the following trigger:
%%
% \IEEEtriggeratref{3}
%%
%% which triggers a \newpage (i.e., new column) just before the given
%% reference number. Note that you need to adapt this if you modify
%% the paper. The "triggered" command can be changed if desired:
%%
% \IEEEtriggercmd{\enlargethispage{-20cm}}
%%
%%%%%%

%%%%%%
%% References:
%% We recommend the usage of BibTeX:
%%
\bibliographystyle{IEEEtran}
\bibliography{exportlist.bib}
%\bibliography{definitions,bibliofile}
%%
%% where we here have assume the existence of the files
%% definitions.bib and bibliofile.bib.
%% BibTeX documentation can be obtained at:
%% http://www.ctan.org/tex-archive/biblio/bibtex/contrib/doc/
%%%%%%
%% Or you use manual references (pay attention to consistency and the
%% formatting style!):



%%%%%% 
%% Appendix:
%% If needed a single appendix is created by
%%
\appendix
%%
%% If several appendices are needed, then the command
%%
% \appendices
%%
%% in combination with further \section-commands can be used.
%%%%%%

The appendix (or appendices) are optional. For reviewing purposes
additional 5~pages (double-column) are allowed (resulting in a maximum
grand total of 10~pages plus one page containing only
references). These additional 5~pages must be removed in the final
version of an accepted paper.


\end{document}

%%% Local Variables:
%%% mode: latex
%%% TeX-master: t
%%% End:
