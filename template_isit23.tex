%% LaTeX Template for ISIT 2023
%%
%% by Stefan M. Moser, June 2022
%% 
%% derived from bare_conf.tex, V1.4a, 2014/09/17, by Michael Shell
%% for use with IEEEtran.cls version 1.8b or later
%%
%% Support sites for IEEEtran.cls:
%%
%% https://www.michaelshell.org/tex/ieeetran/
%% https://moser-isi.ethz.ch/manuals.html#eqlatex
%% https://www.ctan.org/tex-archive/macros/latex/contrib/IEEEtran/
%%

\documentclass[conference,a4paper]{IEEEtran}


%% depending on your installation, you may wish to adjust the top margin:
\addtolength{\topmargin}{9mm}
%% apart from this
%% *** do not adjust lengths that control margins, column widths, etc.! ***
%% *** do not use packages that alter fonts (such as pslatex)!          ***

%%%%%
%% Packages:
\usepackage[utf8]{inputenc} 
\usepackage[T1]{fontenc}
\usepackage{url}              % provides \url{...}
%\usepackage{ifthen}          % provides \ifthenelse
\usepackage{cite}             % improves presentation of citations

\usepackage[cmex10]{amsmath}  % Use the [cmex10] option to ensure complicance
                              % with IEEEXplore (see bare_conf.tex)
\interdisplaylinepenalty=1000 % As explained in bare_conf.tex
\usepackage{mleftright}       % fix to wrong spacing of \left-,
\mleftright                   % \middle- \right-commands 

\usepackage{graphicx}         % provides \includegraphics{...} to
                              % include graphics (pdf format)
\usepackage{booktabs}         % fixes poor spacing in tables and
                              % provides \toprule, \midrule, \bottomrule

%\usepackage{algorithmicx}    % provides an algorithmic environment for
                              % describing algorithms. See
                              % https://ctan.org/pkg/algorithmicx

% \usepackage[caption=false,font=footnotesize]{subfig}
                              % provides subnumbering within a
                              % floating figure or table

%% For arrays and multiple-line equations, use the
%% IEEEeqnarray-environment. See
%%              https://moser-isi.ethz.ch/manuals.html#eqlatex  
%% for instructions.

%% Do NOT use amsthm or hyperref!
%% -IEEEtran provides its own versions of theorems.
%% -IEEEXplore does not accept submissions with hyperlinks


%%%%%
%% correct bad hyphenation here
\hyphenation{op-tical net-works semi-conduc-tor}
\usepackage{amsthm}
\usepackage{amsfonts}
\DeclareMathOperator{\dist}{dist}
\def\E{\mathbb{E}}
\def\R{\mathbb{R}}
\def\d{\mathrm{d}}
\newtheorem{example}{Example}
\newtheorem{definition}{Definition}
\newtheorem{theorem}{Theorem}
% -------------------------------------------------------------------------
\begin{document}

\title{On the Sample Complexity of High Dimensional Learning with Interpolation of Dataset}

%%%%%%
\author{%
  \IEEEauthorblockN{Anonymous Authors}
  %\IEEEauthorblockA{%
  %  Please do NOT provide authors' names and affiliations\\
  %  in the paper submitted for review, but keep this placeholder.\\
  %  ISIT23 follows a \textbf{double-blind reviewing policy}.}
}

%%%%%% Please only add the author names and affiliations for the FINAL
%%%%%% version of the paper, but NOT for the paper submitted for review!
%
%%%%%
%%%%% Single author, or several authors with same affiliation:
% \author{%
%   \IEEEauthorblockN{Stefan M.~Moser}
%   \IEEEauthorblockA{ETH Zürich\\
%                     8092 Zürich, Switzerland\\
%                     moser@isi.ee.ethz.ch}
%                   }
%
%%%%%
%%%%% Several authors with up to three affiliations:
% \author{%
%   \IEEEauthorblockN{Stefan M.~Moser}
%   \IEEEauthorblockA{ETH Zürich\\
%                     ISI (D-ITET), ETH Zentrum\\
%                     8092 Zürich, Switzerland\\
%                     moser@isi.ee.ethz.ch}
%   \and
%   \IEEEauthorblockN{Albus Dumbledore and Harry Potter}
%   \IEEEauthorblockA{Hogwarts School of Witchcraft and Wizardry\\
%                     Hogwarts Castle\\ 
%                     1714 Hogsmeade, Scotland\\
%                     \{dumbledore, potter\}@hogwarts.edu}
% }
%
%%%%%   
%%%%% Many authors with many affiliations:
% \author{%
%   \IEEEauthorblockN{Albus Dumbledore\IEEEauthorrefmark{1},
%                     Olympe Maxime\IEEEauthorrefmark{2},
%                     Stefan M.~Moser\IEEEauthorrefmark{3}\IEEEauthorrefmark{4},
%                     and Harry Potter\IEEEauthorrefmark{1}}
%   \IEEEauthorblockA{\IEEEauthorrefmark{1}%
%                     Hogwarts School of Witchcraft and Wizardry,
%                     1714 Hogsmeade, Scotland,
%                     \{dumbledore, potter\}@hogwarts.edu}
%   \IEEEauthorblockA{\IEEEauthorrefmark{2}%
%                     Beauxbatons Academy of Magic,
%                     1290 Pyrénées, France,
%                     maxime@beauxbatons.fr}
%   \IEEEauthorblockA{\IEEEauthorrefmark{3}%
%                     ETH Zürich, ISI (D-ITET), ETH Zentrum, 
%                     CH-8092 Zürich, Switzerland,
%                     moser@isi.ee.ethz.ch}
%   \IEEEauthorblockA{\IEEEauthorrefmark{4}%
%                     National Yang Ming Chiao Tung University (NYCU), 
%                     Hsinchu, Taiwan,
%                     moser@isi.ee.ethz.ch}
% }
%

\maketitle

%%%%%
%% Abstract: 
%% If your paper is eligible for the student paper award, please add
%% the comment "THIS PAPER IS ELIGIBLE FOR THE STUDENT PAPER
%% AWARD." as a first line in the abstract. 
%% For the final version of the accepted paper, please do not forget
%% to remove this comment!
%%
\begin{abstract}
  This paper estimates the sample complexity for a kind of efficient learning in high dimensional space.
  To be specific, we model the learning task as interpolation of convex hull consisting of i.i.d. sampled data
  , and the sample complexity is the number of training data which make the probability measure of the convex hull
  tend to one. It is shown that the sample complexity has exponential relationship with the dimension parameter for all the distribution families
  considered in this paper, which gives insight on why high dimensional learning is difficult in a general sense. 
\end{abstract}


\section{Introduction}
\label{sec:intro}
In machine learning, intuitively we model a learning task as interpolating the
training data. Under such interpretation, we are prone to think that state-of-the-art
algorithms work well in high dimensional space because the models can interpolate training data well. 
However, Balestriero et al. \cite{balestriero2021learning}
argues that this is a misconception and the interpolation almost surely never occurs in high-dimensional spaces.
In the terminology of Balestriero, the interpolation error is made whenever a sample does not belong to the convex hull of the dataset.

Inspired by Balestriero, we investigate under which condition the interpolation error converges to zero,
as the sample size and the dimension tend to infinity simultaneously. In our work, we give
an asymptotic upper bound of the interpolation error and obtain the sample complexity of this kind of
high dimensional learning. We show that when the training data are sampled from polynomial, exponential and truncated distribution families,
we need exponentially large number of samples for efficient learning. Our result is consistent with the common sense
that direct learning
without dimension reduction is impractical. Besides the relationship between $N$ and $d$,
we also reveal the dependency between $N$ and the decaying rate of the distribution.

The study of interpolation error $p_{N,d}$ is closely related with the quantity $\E[F_N]$,
which is the expectation of the number of facets of the convex hull. Loosely speaking,
the inequality relationship $p_{N,d} \leq \frac{\E[F_N]}{Nd}$ holds. Therefore, the asymptotic
value of $\E[F_N]$ gives an upper bound of $p_{N,d}$.
The study of $\E[F_N]$
is an old topic in stochastic geometry. The recent advance dates to Dwyer's work on the estimation of $\E[F_N]$,
in which the order of $\E[F_N]$ about $N$ is obtained \cite{dwyer1991convex}. In our paper, we will extend Dwyer's work
to get a more accurate asymptotic value of $\E[F_N]$ in $\R^d$.

The organization of this paper is as follows: In Section \ref{sec:int_f}, we develop a method of obtaining $\E[F_N]$ by generalizing Carnal's integration formula
% of $H(x)=P(d_{12}\geq x)$ 
from $d=2$ to higher dimensions.
In Section \ref{sec:three_distriutions}, we then derive the explicit asymptotic
expressions of $\E[F_N]$ and the sufficient condition for $p_{N,d}\to 0$ as $N,d \to \infty$
for three different distribution families.
The main contribution of this paper, which is summarized in Section \ref{sec:conclusion},
is to provide a complete integration formula for $\E[F_N]$,
and following the classification of distributions of previous authors \cite{carnal1970konvexe,dwyer1991convex},
we derive the asymptotic expression
of $\E[F_N]$ for each category.

Below we define some notations to be used throughout this paper.
Without specific emphasis, all $d$-dimensional distributions considered in this paper are spherically symmetric.
Let $X=(X^{(1)},\dots, X^{(d)})$ follow such a distribution.
Then we use $F(x)=P(|X|\geq x)$ to denote the probability that a random point lies outside
a $d$-sphere ($d$-dimensional sphere centered at origin) with radius $x$.
The marginal distribution of each component of $X$ is determined by $G(x)=P(X^{(1)}\geq x)$.
Let $dist$ represent the distance from the origin to the hyperplane spanned by $X_1, \dots, X_d$.
Specifically, in 2-D, the hyperplane is the straight line passing through $X_1$ and $X_2$.
We use $H(x)=P(\dist\geq x)$ to denote the probability that $\dist$ is larger than $x$.
When $n$ tends to infinity, the functions $f(n)$ and $g(n)$ are asymptotic equivalent if $\lim_{n\to \infty} \frac{f(n)}{g(n)}=1$.
We denote this relationship as $f(n) \sim g(n)$.
\section{Related work}

The asymptotic expression of $\E[F_N]$ as $N\to \infty$
was studied initially by R{\'e}nyi and Sulanke \cite{renyi1963konvexe}.
They considered
bivariate Gaussian distribution and uniform distribution
in a planar convex region.
Later on,  R{\'e}nyi's work was generalized by
Carnal \cite{carnal1970konvexe}, who
classified symmetric 2-D distributions
into three categories according to their tails:
polynomial, exponential, and truncated tails.
Then he obtained the asymptotic expression of $\E[F_N]$
for each category of distributions.




The study of $\E[F_N]$ for $d>2$ was made firstly by
Raynaud
\cite{raynaud1970enveloppe}.
He obtained the asymptotic formula of $\E[F_N]$
for uniform distribution in a hyperball
and standard Gaussian distribution in $\mathbb{R}^d$.
Afterwards, following Carnal, Dwyer \cite{dwyer1991convex}
estimated the order of $\E[F_N]$ about $N$
for three different distribution families.

Below we list some other related works which are not mentioned in the previous section.
For $d=2,3$, Efron \cite{efron1965convex} obtained the explicit integration formula for $\E[F_N]$ and $p_{N,d}$ 
when samples follow Gaussian distribution or uniform distribution within a unit sphere.

Davis et al. \cite{davis1987convex} found the relationship between the distributions with algebraic tails $F(x) \sim x^{-k}$ and Poisson random process.
They showed that for $k>0, d=2$,
$\mathrm{H}_N$ converges to the convex hull of a Poisson random process,
and
the limit of $\E[F_N]$ can be derived from this random process.
The special case $k=0, d=2$ is treated in a later paper, in which the
limit distribution of $F_N$ is computed \cite{aldous1991number}.

Studies related to distributions with exponential tails mainly focus on Gaussian distribution.
Kabluchko et al. \cite{kabluchko2020absorption} obtained explicit expressions for $p_{N,d}$;
Affentranger \cite{affentranger1991convex} derived $\E[F_N]$;
Hueter et al. \cite{hueter1999limit} studied the concentration property of $F_N$ and obtained
an upper bound for $\E[F_N]$ sharper than the bound in \cite{dwyer1991convex}.

For truncated tails,
Affentranger \cite{affentranger1991convex} studied one sub-category called beta-typed distribution and obtained
the asymptotic value of $\E[F_N]$.

Beside the number of facets $\E[F_N]$,
the asymptotic value of other quantities related with $\mathrm{H}_N$, such as the volume, and the surface area,
were systematically investigated in a general framework involving the property of facets
\cite{schneider2008stochastic, barany2008random}.

\section{Mathematic Models}
\label{sec:int_f}

Let $\mathcal{X}_N = \{X_1, X_2, \dots, X_N\}$ be a set of i.i.d. random points generated from
a spherically symmetric distribution in $\mathbb{R}^d$.
Let $\mathrm{H}_N$ denote the convex hull of $\mathcal{X}_N$,
and the interpolation error is defined as $p_{N,d}=P(X_{N+1} \not\in \mathrm{H}_N)$.

To study $p_{N,d} \to 0$, we introduce two additional quantities of $H_N$: the number of vertices
$V_N$ and the number of facets $F_N$.
A facet of a $d$-dimensional object is one of its $(d-1)$-dimensional faces.
Besides, 
In this paper, we study the mathematical expectation
of $F_N$, denoted as $\E[F_N]$, and we consider its asymptotic value as $N\to \infty$.

The starting point is the integration formula for $\E[F_N]$.
For $d\geq 2$ and $N\geq d+1$, the formula is given as
\begin{align}
     \E[F_N] &= \binom{N}{d} \int_0^{\infty} 
     \left[G(x)^{N-d} + (1-G(x))^{N-d} \right]|\mathrm{d} H(x)| 
    \label{eq:E_F_N_d}
\end{align}
Formula \eqref{eq:E_F_N_d} first appeared in the proof section of \cite{raynaud1970enveloppe}
and can further be generalized to compute other combinatorial properties of $\mathrm{H}_N$ (like surface area and volume)
\cite{barany2008random}.
Dwyer obtained the integration formula for $G(x)$ in $d\geq 2$ as:
\begin{align}\label{eq:G_d_kappa}
     G(x) & = \int_x^{+\infty} \kappa \left(\frac{x}{y} \right) |\mathrm{d}F(y)| \\
     \kappa(r) & = \frac{\Gamma(\frac{d}{2})}
     {\sqrt{\pi}\Gamma(\frac{d-1}{2})}\int_r^{1}
     (1-u^2)^{(d-3)/2}\mathrm{d}u\label{eq:kappa_r}
\end{align}
where $\kappa(r)$ is the fraction of the surface area of a unit $d$-sphere
cut off by a plane at distance $r$ from the origin.

We cannot compute $\E[F_N]$ from $F(x)$ yet, since no formula for $H(x)$ with respect to $F(x)$ is given in previous studies.
Our main theorem solves this problem by giving the integration formula for $H(x)$ in $d\geq 2$:
\begin{theorem}\label{thm:H}
For $d\geq 2$, the integration formula for $H(x)$ is given as
\begin{equation}
     H(x) = \frac{2}{\pi}
     \int_x^{+\infty} \arccos\frac{x}{y}
     |\mathrm{d} (K^d(y))|\label{eq:H_expression_d_dim}
\end{equation}
where the auxiliary function $K(x)$ is defined in the following way:
\begin{align}
     \lambda_d(x)  :&=(1-x^2)^{\frac{d-2}{2}}
     \label{eq:lambda_r}\\
     K(x) :&=P\left(\sqrt{(X^{(1)})^2+(X^{(2)})^2}>x \right)\notag \\
     &=
     \int_x^{+\infty} 
     \lambda_d \left(\frac{x}{y} \right)|\d F(y)|
     \label{eq:K_x}
\end{align}
\end{theorem}
%It is easy to observe that \eqref{eq:H_expression_d_dim} reduces to 
%\eqref{eq:H_expression_2_dim} when $d=2$.
For standard Gaussian distribution in $\R^d$,
$X_1, X_2$ are independent
Gaussian random variables, and we can verify that \eqref{eq:K_x} holds with $K(x) = e^{-x^2/2}$.
In fact, $1-K(x)$ is the CDF of Rayleigh distribution.
%For general spherical distributions, the function $K(x)$ itself is irreverent with the dimension $d$.
The proof of Theorem \ref{thm:H}
is provided in Appendix \ref{app:th}.
\section{Three distribution families}\label{sec:three_distriutions}
Notice that in \eqref{eq:E_F_N_d}, $G(x)<\frac{1}{2}$ for $x>0$. Therefore the
term $G(x)^{N-d}$ decays at an exponential rate as $N-d\to \infty$.
As a result, we obtain
\begin{align}
     \E[F_N] \sim \binom{N}{d} \int_0^{+\infty} 
      (1-G(x))^{N-d} |\d H(x)| \textrm{ as } N-d\to \infty
     \label{eq:E_F_N_d_sim}
\end{align}
Let $\E[V_N]$ represent the expected number of
vertices of $\mathrm{H}_N$.
By far it is not clear whether there exist explicit relations
between $\E[F_N]$ and $\E[V_N]$
for general spherical symmetric distributions.
However, we can give some estimations by inequality.
Using Corollary 19.6 of \cite{brondsted2012introduction}, we have the following
inequality:
\begin{equation}\label{eq:F_V_upper}
     F_N \geq (d-1) V_N - (d+1)(d-2)
 \end{equation}
Combined with $p_{N,d} = \frac{\E[V_{N+1}]}{N+1}$, which comes from
\cite{efron1965convex}, we have the following upper bound for $p_{N,d}$:
\begin{equation}\label{eq:p_N_d_bound}
    p_{N,d} \leq \frac{\E[F_N]}{d N} \textrm{ as } N \gg d
\end{equation}

We use the symbol $a \gg b$ if $a$ is a function of $b$ and $\lim_{b\to \infty} \frac{a}{b} = \infty$.
Based on \eqref{eq:E_F_N_d_sim}, in the following
we derive the asymptotic expression of $\E[F_N]$
for three different distribution families.

To precisely define these distribution families, we introduce the concept of slowly varying function.
\begin{definition}
A function $L(x)$ is
slowly varying as $x\to \infty$
if for all $\lambda>0$,
$\lim_{x\to\infty}\frac{L(\lambda x)}{L(x)}=1$
holds.
\end{definition}

\subsection{Distributions with polynomial tails}

With the help of a slowly varying function $L(x)$,
distributions with polynomial tails are written
in the following form:
\begin{equation}\label{eq:F_poly_tail}
     F(x) = x^{-k} L(x), k\geq 0
\end{equation}

Dwyer\cite{dwyer1991convex} has obtained $G(x)$ as:
\begin{equation}\label{eq:g_poly_tail}
     G(x) \sim \frac{\Gamma(\frac{d}{2})}{2\sqrt{\pi} \Gamma(\frac{d-1}{2})}
     B\left(\frac{k+1}{2}, \frac{d-1}{2}\right) F(x)  \textrm{ as } x\to \infty
\end{equation}

From Theorem \ref{thm:H}, we obtain the asymptotic
expression of $H(x)$ and $\E[F_N]$, which is
summarized in the following theorem:
\begin{theorem}\label{thm:poly_tails}
     For distributions with polynomial tails defined in \eqref{eq:F_poly_tail},
     we have
\begin{equation}\label{eq:H_poly_tail_exp}
     H(x) \sim \frac{2^d \pi^{(d-1)/2}\Gamma^d(\frac{k}{2}+1)
     \Gamma(\frac{dk+1}{2})}{
         \Gamma^d(\frac{k+1}{2}) \Gamma(\frac{dk}{2}+1)} G(x)^d 
         \textrm{ as } x\to \infty
\end{equation}
and 
\begin{equation}\label{eq:efn_poly_second_deri}
    \E[F_N] \sim \frac{2^d \pi^{(d-1)/2}\Gamma^d(\frac{k}{2}+1)
    \Gamma(\frac{dk+1}{2})}{
        \Gamma^d(\frac{k+1}{2}) \Gamma(\frac{dk}{2}+1)}
        \textrm{ as } N \to \infty, d \textrm { is fixed}
\end{equation}
\end{theorem}
Equation \eqref{eq:efn_poly_second_deri} tells
us that $\E[F_N]$ converges to a constant as $N \to \infty$.
Without consideration of the actual constant, the result of
Theorem \ref{thm:poly_tails} is an refinement to Dwyer's Theorem 1 for the expected number of facets \cite{dwyer1991convex}.
When $d=2$, the result was obtained in (2.4) of \cite{carnal1970konvexe}
and Theorem 4.4 of \cite{davis1987convex}.
\begin{example}
     We consider a special case of polynomial tail called multivariate t-distribution.
     Its pdf is given by
     \begin{equation}\label{eq:pxy_student_t}
          p(x) = \frac{\Gamma((k+d)/2)}{\Gamma(k/2)(k\pi)^{d/2}}
          \left(1+\frac{1}{k}||x||^2
          \right)^{-\frac{k+d}{2}}, x \in \mathbb{R}^d
      \end{equation}
      The parameter $k$ is the degree of freedom for this distribution.
      We can obtain the radius distribution
      $F(x)$ as $F(x) \sim \frac{2\Gamma(\frac{k+d}{2})}{\Gamma(k/2)\Gamma(d/2)} k^{k/2-1} x^{-k}$.
      The marginal distribution is Student's t-distribution. To obtain $G(x)$, which is just the tail area
of the t-distribution, we use an existing asymptotic result found in \cite{andrew1976}.
\begin{equation} \label{eq:eq_dv}
    G(x) \sim k^{\frac{k}{2}-1} \frac{\Gamma \left(\frac{k+1}{2} \right)}
    {\sqrt{\pi} \Gamma\left(\frac{k}{2}\right)}x^{-k}
\end{equation}
After simplification, \eqref{eq:eq_dv} is the same as \eqref{eq:g_poly_tail}.
On the other hand, using similar techniques as (1.4) in \cite{raynaud1970enveloppe},
we can obtain the pdf of the distance of the hyperplane to the origin as
\begin{equation}\label{eq:dH_dx_t_distribution}
     \frac{\d H(x)}{\d x} =  \frac{2}{\sqrt{k} B(\frac{kd}{2},\frac{1}{2})} \left(1 + \frac{x^2}{k} \right)^{-\frac{kd+1}{2}} 
\end{equation}
From above, we get the asymptotic relation $H(x) \sim \frac{2 \Gamma(\frac{kd+1}{2}) k^{kd/2-1}}{d\sqrt{\pi} \Gamma(\frac{kd}{2})}
x^{-kd}$ as $x\to \infty$, which is equivalent to \eqref{eq:H_poly_tail_exp}.

\end{example}
When we allow $d\to \infty$, we treat $N$ as a function of $d$.
If the condition $N/d^2 \to \infty$ is satisfied, we have
\begin{equation}\label{eq:poly_E_F_N_d_infty}
\E[F_N] \sim \sqrt{\frac{2}{\pi dk}}\left(
      \frac{\sqrt{\pi}k \Gamma(k/2)}
     {\Gamma(\frac{k+1}{2})}
 \right)^d \textrm{ for } k>0
\end{equation}
Then the sufficient condition for $p_{N,d} \to 0$ is obtained
from the estimation inequality \eqref{eq:p_N_d_bound},
combined with the asymptotic expressions for polynomial distribution families in \eqref{eq:poly_E_F_N_d_infty}.
\begin{theorem}
  For distributions with polynomial tails with $k>0$,
  when the following condition is satisfied
\begin{equation}
  N \gg \frac{c^d}{d^{3/2}}, \textrm{ where } c=\frac{\sqrt{\pi}k\Gamma(k/2)}{\Gamma(\frac{k+1}{2})}>1  
\end{equation}
the interpolation error $p_{N,d} \to 0$ as $N,d\to \infty$.
\end{theorem}
From the above theorem, we see that $N$ has exponential relationship with $d$.
The higher the dimension is, the more samples are needed.
Besides, the decaying rate of the distribution also has influence on the sample size $N$.
Since $c$ is an increasing function with $k$,
larger $k$ (faster decay) also leads to more samples.

\subsection{Distributions with exponential tails}
In this subsection, we first give the definition of distributions with exponential tails.
\begin{definition}
  For a radius distribution $F(x)$, if we could find
  a slowly varying function $L(x)$ such that
  $x = L(1/F(x))$, we say that the distribution has exponential tails.
\end{definition}
Following Carnal \cite{carnal1970konvexe},
we define the following utility functions:
\begin{align}
     \epsilon(s) & = s (\log (L(s)))' \label{eq:epsilon_s}\\
     v(u) &= -\frac{1}{u} \frac{1}{(\log F(u))'}    
\end{align}
The prime notation represents the symbol for differentiation. When $v(u)$ satisfies
certain regularity conditions prescribed in (2.15) of \cite{carnal1970konvexe},
we can obtain that $\epsilon(s)$ is a slowly varying function.
Dwyer \cite{dwyer1991convex} obtains the expression of $G(x)$ as:
\begin{equation}\label{eq:G_x_exp}
     G(x) \sim \frac{2^{(d-3)/2}}{\sqrt{\pi}}\Gamma\left(\frac{d}{2}\right)
     v^{(d-1)/2}(x) F(x)
      \textrm{ as } x\to \infty
\end{equation}
We make the analysis complete by giving the following theorem.
\begin{theorem}\label{thm:exponential_tails}
     Suppose there exists a slowly
     varying function $L(x)$ such that \linebreak $x=L(1/F(x))$,
     and $\epsilon(s)$ is defined in \eqref{eq:epsilon_s}, then
\begin{equation}\label{eq:H_x_exp}
     H(x) \sim \frac{\pi^{\frac{d-1}{2}} 2^{\frac{d+1}{2}}}{\sqrt{d}}v^{\frac{-d+1}{2}}(x)G^d(x)
     \textrm{ as } x\to \infty
\end{equation}
 and
 \begin{equation}\label{eq:exp_e_f_n}
     \E[F_N]\sim \frac{\pi^{\frac{d-1}{2}} 2^{\frac{d+1}{2}}}{\sqrt{d}} (\epsilon(N))^{-\frac{d-1}{2}}
     \textrm{ as } N \to \infty, d \textrm { is fixed}
 \end{equation}
\end{theorem}
 Using \eqref{eq:epsilon_s}, we note that the term
 $(\epsilon(N))^{-\frac{d-1}{2}}$
 is obtained in Dwyer's Theorem 2 \cite{dwyer1991convex}.
 Our improvement on the asymptotic value of $\E[F_N]$ is that we obtain its preceding term only involving $d$.
 When $d=2$, equation \eqref{eq:exp_e_f_n} is consistent with (2.20) of \cite{carnal1970konvexe}.
 For standard $d$-dimensional Gaussian distribution, we have $L(s)=\sqrt{2\log s}$.
 From \eqref{eq:epsilon_s}, $\epsilon(s) = (2\log s)^{-1}$. And we obtain from \eqref{eq:exp_e_f_n}
 that $\E[F_N]\sim \frac{2^d}{\sqrt{d}}(\pi \log N)^{\frac{d-1}{2}}$,
 which is consistent with (1.11) of \cite{raynaud1970enveloppe}.

 If $d\to\infty$ and $N/d^2\to \infty$, we have
\begin{align}\label{eq:d_infty_exp_E_F_N}
      \E[F_N]\sim \frac{\pi^{\frac{d-1}{2}} 2^{\frac{d+1}{2}}}{\sqrt{d}} \epsilon(N)^{\frac{-d+1}{2}}
\end{align}

Similar to the analysis of polynomial tails, we obtain the
sufficient condition for $p_{N,d} \to 0$ from
\eqref{eq:p_N_d_bound} and \eqref{eq:d_infty_exp_E_F_N}.
\begin{theorem}\label{thm:exp_tails_sample}
  For distributions with exponential tails, when the following condition is satisfied
  \begin{equation}
    N\cdot \epsilon(N)^{(d-1)/2} \gg \frac{(2\pi)^{d/2}}{d^{3/2}},
  \end{equation}
  the interpolation error $p_{N,d} \to 0$ as $N,d\to \infty$.
\end{theorem}
For the special case of exponential distribution
of the form $F(x) \sim C(k)\exp(-x^k)$, we obtain from \eqref{eq:epsilon_s}
that $\epsilon(s)=(k\log s)^{-1}$.
Then from Theorem \ref{thm:exp_tails_sample},
we can estimate that
$N(d)$ grows faster than $c^d$ for any constant $c$ but slower than $\exp(d^2)$.
Besides, $N$ increases with $k$ for fixed $d$.
In other words, larger $k$ (faster decay) leads to more samples.

\subsection{Distributions with truncated tails}
In this subsection, let $L(x)$ also represent a slowly varying function, and we consider the distribution which satisfies
the following conditions:
\begin{equation}\label{eq:F_truncated}
     F(1-x) \sim x^k L\left(\frac{1}{x} \right)  \text{ as } x \to 0^+, k> 0,
     F(x) = 0 \text{ for } x \geq 1
\end{equation}
In \eqref{eq:F_truncated}, $x \to 0^+$ means that $x\to 0$ from the right hand side of $0$.

As mentioned by Dwyer \cite{dwyer1991convex}, uniform distribution
in the unit ball satisfies \eqref{eq:F_truncated} with
$F(1-x) \sim d\cdot x$.

After simplification of (4.1) in \cite{dwyer1991convex}, we obtain
\begin{align}
    G(1-x) \sim a
    L\left(\frac{1}{x} \right)
    x^{k+\frac{d-1}{2}} \textrm{ as } x \to 0^+ 
    \label{eq:truncated_G_1_x}
\end{align}
where
\begin{align}
a &=\frac{2^{\frac{d-1}{2}} k \Gamma(\frac{d}{2})}
    {(d-1) \sqrt{\pi} \Gamma(\frac{d-1}{2})}
    B\left(k, \frac{d+1}{2}\right) \notag \\
    &= \frac{2^{\frac{d-3}{2}} k \Gamma(\frac{d}{2})\Gamma(k)}
    {\sqrt{\pi} \Gamma\left(k+\frac{d+1}{2}\right)}
    \label{eq:a}
\end{align}
Below, we give the theorem in regards to the asymptotic value of $H(1-x)$ and $\E[F_N]$:
\begin{theorem}\label{thm:truncated_tails}
     For distributions with truncated tails
     defined in \eqref{eq:F_truncated},
     we have
\begin{align}
     H(1-x)  \sim b
     L^d(1/x) x^{d(k+\frac{d}{2}-1)+\frac{1}{2}} 
     \textrm{ as } x \to 0^+ \label{eq:truncated_H_1_x}
\end{align}
where
\begin{align}
     b =  \frac{k^d}{\pi}
     2^{\frac{1}{2} + d(\frac{d}{2}-1)} B^d\left(k, \frac{d}{2}\right)
     B\left( \frac{1}{2},
     d\left(k+\frac{d}{2} -1 \right)+1 \right)
     \label{eq:b}
 \end{align}
When $N\to \infty$ and $d$ is fixed, we have
 \begin{align}\label{eq:efn_truncated_formula}
     \E[F_N] \sim &\frac{b}{d!}a^{-d+\frac{d-1}{2k+d-1}}
     \Gamma 
     \left(d+1-\frac{d-1}{2k+d-1}\right)
     \notag\\
     &\cdot N^{\frac{d-1}{2k+d-1}}
     L(N)
     ^{\frac{d-1}{2k+d-1}}
 \end{align}
 where the parameter $a$ is defined in \eqref{eq:a}.
\end{theorem}
 The leading polynomial term $N^{\frac{d-1}{2k+d-1}}$ in \eqref{eq:efn_truncated_formula}
 is the same with that in Dwyer's Theorem 3
 but other coefficients are missing in Dwyer's result
 \cite{dwyer1991convex}. Therefore, we say that Dwyer only made the order estimation
 of $\E[F_N]$ while we obtain its asymptotic value.
 When $d=2$, equation \eqref{eq:efn_truncated_formula} reduces to (3.4) of \cite{carnal1970konvexe}.
 For uniform distribution with $k=1, L(N)=d$, we can verify that
 \eqref{eq:efn_truncated_formula} is equivalent with (1.1)
 of \cite{raynaud1970enveloppe}.
 \begin{example}
  We consider a special distribution with truncated tail called
  beta-typed distribution. Its density function is defined as:
  \begin{equation}\label{eq:pxy_beta_t}
    p(x) = \frac{\Gamma(\frac{d}{2}+ k)}{\pi^{d/2} \Gamma(k)} (1-||x||^2)^{k-1},
    ||x|| <1, x \in \mathbb{R}^d
\end{equation}
 $p(x)=0$ for $||x||\geq 1$.
 The so-called beta-typed distribution first appeared in \cite{affentranger1991convex}.
 In Equation (3.1) of Affentranger's formulation,
 he uses the symbol $q$, which has the relationship  $k=q+1$ with our symbol $k$.
 From \eqref{eq:pxy_beta_t} and \eqref{eq:F_truncated}, we can obtain $L(N)=\frac{2^k}{kB(d/2,k)}$.
 Besides, similar to \eqref{eq:dH_dx_t_distribution},
 we can obtain the pdf of the distance of the hyperplane to the origin as
\begin{equation}
     \frac{\d H(x)}{\d x} = \frac{2}{B(\frac{1}{2}, d(k+\frac{d}{2}-1)+\frac{1}{2})}\left(1 -x^2\right)^{d(k+\frac{d}{2}-1)-\frac{1}{2}} 
\end{equation}
From above, we get the asymptotic relation $H(x) \sim \frac{2^{\mu}}{\mu B(1/2, \mu)}
x^{\mu}$ with $\mu=d(k+\frac{d}{2}-1)+\frac{1}{2}$ as $x\to \infty$,
which is equivalent to \eqref{eq:truncated_H_1_x}.

 We can further verify that
 \eqref{eq:efn_truncated_formula} is equivalent with the expression of $c_3$
 in (3.3) of \cite{affentranger1991convex}.
 \end{example}
 When we allow $d\to \infty$ and $N/d^2 \to \infty$, we obtain
 \begin{equation}\label{eq:truncated_d_inf}
  \E[F_N] \sim 2^{\frac{d+2k}{2}}\pi^{\frac{d-2}{2}} k\Gamma(k)e^k d^{\frac{d-3}{2}-k}
  N^{\frac{d-1}{2k+d-1}} L(N)^{\frac{d-1}{2k+d-1}}
 \end{equation}

 Similar to the previous analysis, from
 \eqref{eq:p_N_d_bound} and \eqref{eq:truncated_d_inf} we obtain
 the following theorem:
 \begin{theorem}\label{thm:truncated_tails_sample}
  For distributions with truncated tails,
  when the following condition is satisfied
  \begin{equation}
    \frac{N^{2k/d}}{L(N)^{\frac{d-1}{2k+d-1}}}
        \gg (2\pi)^{d/2}d^{(d-5)/2},
  \end{equation}
  the interpolation error $p_{N,d} \to 0$ as $N,d\to \infty$.
\end{theorem}
 
Generally speaking, the sample complexity of $N$ is the smallest for algebraic tails and the largest
for truncated tails. In other words, the faster $F(x)$ decays, the larger $N$ becomes.

%%%%%%
%% An example of a floating figure using the graphicx package.
%% Note that \label must occur AFTER (or within) \caption.
%% For figures, \caption should occur after the \includegraphics.
%%
% \begin{figure}[htbp]
%   \centering
%   \includegraphics[width=0.3\textwidth]{myfigure}
%   % where a .pdf suffix will be assumed for pdflatex
%   \caption{Simulation results.}
%   \label{fig:sim}
% \end{figure}
%%%%%%

%%%%%%
%% An example of a double column floating figure using two subfigures.
%% (The subfig package must be loaded for this to work; uncomment the
%% corresponding line in the document header.)  The *-version of
%% figure makes the figure use both columns. The subfigure \label
%% commands are set within each subfloat command, the \label for the
%% overall figure must come after \caption.  \hfil must be used as a
%% separator to get equal spacing.
%%
% \begin{figure*}[htbp]
%   \centering
%   \subfloat[Caption of first subfigure.]{%
%     \label{fig:subfigA}
%     \includegraphics[width=4cm]{subfigcase1}}
%   \hfil
%   \subfloat[Caption of second subfigure.]{%
%     \label{fig:subfigB}
%     \includegraphics[width=4cm]{subfigcase1}}  
%   \caption{Two figures showing something.}
%   \label{fig:subfigures}
% \end{figure*}
% We refer to the Figures~\ref{fig:subfigA} and \ref{fig:subfigB} within
% Figure~\ref{fig:subfigures}.
%%%%%% 

%%%%%%
%% An example of a floating table.
%% Note that the \caption command should come BEFORE the table. Table
%% text will default to \footnotesize as IEEE normally uses this
%% smaller font for tables.  The \label must come after \caption as
%% always. The following example relies on booktabs.sty to fix poor
%% spacing within tables and using \toprule, \midrule, and \bottomrule
%% as proper replacement for the corresponding \hline
%%
% \begin{table}[htbp]
%   \centering
%   \caption{Recursion given by Lloyd's algorithm for the case of a
%     Gaussian random variable represented by four values.}
%   \label{tab:gaussfour}
%   \begin{IEEEeqnarraybox}[\IEEEeqnarraystrutmode%
%     % \IEEEeqnarraystrutsizeadd{2pt}{1pt}% uncomment for more spacing
%     ]{l"l"l}
%     \toprule
%     \hat{x}_1 & \hat{x}_2 & \theta \\
%     \midrule
%     0.5 & 1 & 0.75 \\      
%     0.3578 & 1.3288 & 0.8433 \\
%     0.3973 & 1.4011 & 0.8992 \\
%     0.4202 & 1.4450 & 0.9326 \\
%     0.4336 & 1.4714 & 0.9525 \\
%     0.4414 & 1.4872 & 0.9643 \\
%     0.4461 & 1.4966 & 0.9714 \\
%     0.4488 & 1.5022 & 0.9755 \\
%     \vdots && \\
%     0.4528 & 1.5104 & 0.9816 \\
%     \bottomrule
%   \end{IEEEeqnarraybox}
% \end{table}
%%%%%%



\section{Conclusion}
\label{sec:conclusion}

We conclude by pointing out that on the last page the columns need to
be balanced. Instructions for that purpose are given in the source
file (they are commented out).

%%%%%%
%% To balance the columns at the last page of the paper use this
%% command somewhere at the top of the first column of the last page:
%%
% \enlargethispage{-5cm} 
%%
%% where the exact amount of page reduction has to be adapted to the
%% actual situation.
%%
%% If the balancing should occur in the middle of the references, use
%% the following trigger:
%%
% \IEEEtriggeratref{3}
%%
%% which triggers a \newpage (i.e., new column) just before the given
%% reference number. Note that you need to adapt this if you modify
%% the paper. The "triggered" command can be changed if desired:
%%
% \IEEEtriggercmd{\enlargethispage{-20cm}}
%%
%%%%%%

%%%%%%
%% References:
%% We recommend the usage of BibTeX:
%%
\bibliographystyle{IEEEtran}
\bibliography{exportlist.bib}
%\bibliography{definitions,bibliofile}
%%
%% where we here have assume the existence of the files
%% definitions.bib and bibliofile.bib.
%% BibTeX documentation can be obtained at:
%% http://www.ctan.org/tex-archive/biblio/bibtex/contrib/doc/
%%%%%%
%% Or you use manual references (pay attention to consistency and the
%% formatting style!):



%%%%%% 
%% Appendix:
%% If needed a single appendix is created by
%%
\appendix
%%
%% If several appendices are needed, then the command
%%
% \appendices
%%
%% in combination with further \section-commands can be used.
%%%%%%

\section{Proof of Theorem \ref{thm:H}}\label{app:th}


\end{document}

%%% Local Variables:
%%% mode: latex
%%% TeX-master: t
%%% End:
